\documentclass[12pt,letterpaper]{article}
\usepackage{amsmath,amsthm,amsfonts,amssymb,amscd}
\usepackage{fullpage}
\usepackage{lastpage}
\usepackage{enumerate}
\usepackage{fancyhdr}
\usepackage{mathrsfs}
\usepackage[margin=3cm,bottom=6cm]{geometry}
\usepackage{wrapfig}
\usepackage{graphicx}

\setlength{\parindent}{0.0in}
\setlength{\parskip}{0.05in}

\renewcommand{\theenumi}{\bf\Alph{enumi}}

% Edit these as appropriate
\newcommand\course{Math 227C}
\newcommand\semester{Spring 2019}     % <-- current semester
\newcommand\hwnum{3}                  % <-- homework number
\newcommand\yourname{Jun Allard} % <-- your name
%\newcommand\login{jcarberr}           % <-- your CS login

\newenvironment{answer}[1]{
  \subsubsection*{Problem \hwnum.#1}
}{\newpage}

\pagestyle{fancyplain}
\headheight 35pt
\lhead{ Math 227C}
\chead{\textbf{ Problem Set 6}}
%\rhead{Due {\bf Friday, May 11th}}
\headsep 20pt

\begin{document}

% \begin{enumerate}[A.] % uncomments for multi-problem homeworks

%%%%%%%%%%%%%% PROBLEM %%%%%%%%%%%%%%%%%%


%%%%%%%%%%%%%% PROBLEM %%%%%%%%%%%%%%%%%%
%\item 
Many processes, including the spread of an infectious disease through a small community, can be modeled as first-order exponential processes like
\begin{equation*}
\frac{dV}{dt} = \frac{1}{\tau}\left(R-1\right) V \quad V(0)=1 \label{eq:exponential}
\end{equation*}
where $V$ is the tumor volume, measured in number of cells, and $R$ is a constant.

This will either lead to exponential growth or exponential decay.

The constant $R$ is different for every patient.
Assume it has Gaussian distribution with mean 1 and standard deviation $\sigma$,
\begin{equation*}
p_R(r) = \frac{1}{\sqrt{2\pi \sigma^2}}\, e^{-\left(r-1\right)^2/2\sigma^2}.
\end{equation*}

\begin{enumerate}[i. ]
\item Find the probability density function $p_V(v,t)$ of $V(t)$.
\end{enumerate}

Intuitively, we expect half of the trajectories to grow exponentially, and half of the trajectories to decay exponentially.

\begin{enumerate}[i. ]
\setcounter{enumi}{1}
\item Sketch or plot the probability density you found for $p_V(v,t)$.
\item What is the probability that a trajectory is above the initial condition at $V=1$? In other words, what is $\mathbb{P}(V(t)>1)$? Is it true that half the trajectories remain above the initial condition $V=1$, and half remain below the initial condition $V=1$? 
\item Suppose $\tau= 1$ months and $\sigma = 0.1$. What percent of patients have a tumor with more than 1000 cells after 10 months?
\end{enumerate}

A slightly more complicated model that is a modified version of Equation~\ref{eq:exponential}, called the Gompertz model, is used to fit patient data. 

%%%%%%%%%%%%%%%%%%%%%%%%%%%%%%%%%%%%%%
% \end{enumerate} % uncomments for multi-problem homeworks
\end{document}

%%%%%%%%%%%%%%%%%%%%%%%%%%%%%%%%%%%%%%
